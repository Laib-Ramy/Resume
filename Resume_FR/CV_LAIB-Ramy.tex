\documentclass{resume}
\usepackage[utf8]{inputenc}
\usepackage[T1]{fontenc}
\usepackage{lmodern} % Utilisation d'une police qui gère bien les caractères accentués
\usepackage{paralist}
\usepackage[a3paper,left=0.9in,top=-0.2in,right=0.6in,bottom=0.5in]{geometry}
\usepackage{pifont}
\usepackage{tabularx}
\usepackage{xcolor}
\usepackage{graphicx}
\usepackage{hyperref}
\usepackage{arydshln}
\usepackage{pagecolor}
\usepackage{tikz}
\usepackage[english]{babel}
\usepackage[inline]{enumitem}

\definecolor{myuniversity}{RGB}{116,28,76}
\definecolor{myblue}{RGB}{146,166,219}
\setlist[enumerate,1]{label=|}
\begin{document}
\noindent\makebox[\textwidth]{
\begin{tikzpicture}[remember picture,overlay]
    
    % Define the gradient from grey to white
    \fill[draw=white,left color=myblue!70,right color=white] (-16, 1) rectangle (\paperwidth , -\paperheight);
    \node[anchor=north west, inner sep=0] at (2, 0){};
\end{tikzpicture}
} 
\customcontactinfo
  {Ramy LAIB} % Name
  {\includegraphics[height=1em]{../Ressource/address.png} 135b Rte de chartres, Bures-sur-Yvette, Île-de-France, France} % Address 
  {\href{mailto:Ramy.laib@proton.me}{\raisebox{-2pt}{\includegraphics[height=1.25em]{../Ressource/email.png}} Ramy.laib@proton.me}}                 % Email
  {\href{https://linkedin.com/in/ramy-laib1618033989}{\includegraphics[height=1.25em]{../Ressource/linkedin.png} linkedin.com/in/ramy-laib/}\phantom{https://linkedin.com/in/ramy-laib1618033989}}  % LinkedIn
  {\href{https://github.com/Laib-Ramy}{\includegraphics[height=1.25em]{../Ressource/github.png} github.com/Laib-Ramy}\phantom{https://github.com/Laib-Ramy}}  % GitHub
  {\href{tel:+33636206685}{\raisebox{-2pt}{\includegraphics[height=1em]{../Ressource/tel.png}} 06 36 20 66 85}}
  %----------------------------------------------------------------------------------------
%	OBJECTIVE
%----------------------------------------------------------------------------------------

  \vspace{-2em}
  \begin{rSection}{Résumé}

    {Ingénieur spécialisé en systèmes et réseaux, passionné par l'innovation technologique. Je souhaite mettre à profit mes compétences en Linux, développement backend (Python, C) et calcul haute performance. Je recherche un poste stimulant dans une organisation dynamique pour contribuer à des projets significatifs et évoluer professionnellement.}
    
    \end{rSection}
    \vspace{-0.75em}
    %----------------------------------------------------------------------------------------
    %	EXPÉRIENCE
    %----------------------------------------------------------------------------------------
    \begin{rSection}{Expérience}
    
        \textbf{Stagiaire Développeur Full Stack} \hfill Février 2024 - Août 2024\\
        \href{https://www.advanthink.com}{\includegraphics[height=1em]{../Ressource/Logo Symbol A en couleur.png}} AdvanThink (isoft)\hfill \textit{Saint-Aubin, FR}
         \begin{itemize}
            \itemsep -6pt {} 
            \item Analyse du système existant et conception d'une architecture orientée services pour l'exécution par lots, améliorant l'efficacité de 60\%.
            \item Développement de scripts DevOps pour l'automatisation de la configuration des services, réduisant le temps de configuration de 50\%.
            \item Rédaction d'une thèse de recherche sur l'optimisation de l'architecture logicielle à l'aide de services et de microservices.
         \end{itemize}
    \end{rSection} 
    \vspace{-0.75em}
    
    %----------------------------------------------------------------------------------------
    %	PROJETS
    %----------------------------------------------------------------------------------------
    \begin{rSection}{Projets}
        \vspace{-0.25em}
        \newcommand\tab[1][1cm]{\hspace*{#1}}
    \begin{itemize}[leftmargin=*,label= ]
        \setlength{\itemsep}{-6pt}
        \item \textbf{\large Conception de Poubelle Intelligente}
        \begin{itemize}
            \setlength{\itemsep}{-6pt}
            \item Solution IoT de tri des déchets médicaux avec \textbf{85 \% de précision}, utilisant Arduino, ESP32, et \textbf{TensorFlowJS} pour l'IA en temps réel.
            \item \textbf{Protocole MQTT} et \textbf{Node-RED} pour une intégration et une communication légères et efficaces.
            \item Projet accessible sur \href{https://github.com/evry-paris-saclay/2023-m2cns-rd-ChaineDeValeurs}{\textbf{GitHub}} pour plus de détails.
        \end{itemize}
            
        
        \item \textbf{\large Implémentation de Réseau de Neurones en C} 
        \begin{itemize}
            \setlength{\itemsep}{-6pt} % Réduction de l'espace entre les items
            \item Développement d'architectures de réseaux neuronaux personnalisées en C pour la classification.
            \item Implémentation de l'initialisation aléatoire des poids et des biais, et de la fonction d'activation sigmoïde.
            \item Conception de la propagation avant/arrière pour l'apprentissage supervisé.
        \end{itemize}
        
        \item \textbf{\large Système de Gestion de Bibliothèque Intégré}
        \begin{itemize}
            \setlength{\itemsep}{-6pt} % Réduction de l'espace entre les items
            \item Conception d'un système de gestion de bibliothèque optimisant de 40\% l'efficacité de recherche de livres.
            \item Développement backend en Python avec Flask, utilisant ProtoBUF pour la sérialisation et Celery pour la planification des tâches.
            \item Gestion d'une base de données robuste de plus de 10 000 livres, assurant l'intégrité des données.
        \end{itemize}
        
        
        \item \textbf{\large Migration d'OracleDB vers MongoDB} 
        \item \begin{itemize}
            \setlength{\itemsep}{-6pt} % Réduction de l'espace entre les items
            \item Migration d'une base de données Oracle sur Oracle Linux 8 vers MongoDB, réduisant les temps de réponse de 50\%.
            \item Développement et exécution de scripts de migration personnalisés assurant une transition sans faille et sans perte de données.
            \item Implémentation de la structure orientée documents de MongoDB pour une meilleure scalabilité et performance des données.
        \end{itemize}
        
        
        \item \textbf{\large Réseau d'Architecture de Campus Résilient}
        \begin{itemize}
            \setlength{\itemsep}{-6pt} % Réduction de l'espace entre les items
            \item Conception et déploiement d'un réseau de campus résilient optimisé pour la haute performance.
            \item Utilisation de VMware pour la virtualisation, Cumulus Linux pour la gestion du réseau, et pare-feux ACL pour la sécurité.
            \item Intégration de l'automatisation avec Ansible et tolérance aux pannes avec LACP et MLAG, améliorant la disponibilité .
        \end{itemize}
        
        \item \textbf{\large Sécurité et Pare-feu sur OpenBSD}
        \begin{itemize}
            \setlength{\itemsep}{-6pt} % Réduction de l'espace entre les items
            \item Administration de la sécurité système et réseau avec déploiement d'un pare-feu Packet Filter sur OpenBSD.
            \item Assurance d'une protection robuste contre les menaces grâce à des mécanismes proactifs de tolérance aux pannes.
            \item Intégration de CARP et PFsync pour synchroniser les tables d'état du pare-feu sur plusieurs nœuds, renforçant la fiabilité du réseau.
        \end{itemize}
        
        \item \textbf{\large Analyse de Données}
        \begin{itemize}
            \setlength{\itemsep}{-6pt} % Réduction de l'espace entre les items
            \item Projet d'analytique avancée en Python, améliorant la précision de classification et modélisation prédictive de 20\%.
            \item Utilisation de techniques avancées comme le clustering, l'analyse de corrélation canonique et l'analyse en composantes principales.
            \item Analyse de jeux de données de plus d'un million d'enregistrements avec pandas et exploration visuelle via matplotlib.
        \end{itemize}
        
    \end{itemize}
    \end{rSection} 
    \vspace{-0.75em}
    \begin{rSection}{Éducation}
    
        {\large \bf Master en Systèmes Informatiques et Réseaux},  \href{https://www.universite-paris-saclay.fr/}{Université Paris-Saclay\includegraphics[height=1.5em]{../Ressource/petitlogo.png}} \hfill {2022-2024}\\
        Cours pertinents : Architecture Orientée Services, Calcul Haute Performance (C, OMP, Pthread, MPI, CUDA), Administration des Systèmes et Réseaux, Systèmes de Gestion de Bases de Données, Recherche Opérationnelle, Science des Données (Python).
        
        {\large \bf Master en Réseaux, Mobilité et Systèmes Embarqués}, \href{https://www.ummto.dz/}{U.M.M.T.O\includegraphics[height=1em]{../Ressource/ummto.png}}\hfill {2022}\\
        Cours pertinents : Sécurité Réseau, Architectures Parallèles et Systèmes Distribués (C, OMP, Pthread, MPI, CUDA), Réseaux Mobiles, Interfaces de Systèmes Embarqués, Assemblage ARM.\\
        Vainqueur du CTF du module Sécurité.
        
        {\large \bf Licence en Informatique : Systèmes Informatiques}, \href{https://www.ummto.dz/}{U.M.M.T.O\includegraphics[height=1em]{../Ressource/ummto.png}}\hfill {2018-2021}\\
        Cours pertinents : Algorithmes, Python, Java, C, Systèmes d'Exploitation, Sécurité des Systèmes d'Information, Développement Web, Développement Mobile, Assemblage ARM.
        Membre du Club d'Informatique.
    \end{rSection}
    \vspace{-0.75em}
    
    \begin{rSection}{Compétences et Talents}
        \vspace{-0.25em}
        \begin{tabularx}{\textwidth}{ @{} >{\bfseries}l @{\hspace{3ex}} X @{} }
        Langages de Programmation & C (HPC, Programmation Parallèle), Python (scikit-learn, Flask, ...), Java, JavaScript \\
        
        Outils & Docker, Ansible, Git, Bash, VMware, VirtualBox, Jenkins, Kubernetes, Terraform, Grafana \\
        
        Bases de Données & OracleDB, MongoDB, SQL \\
        
        Développement Web & HTML, CSS, JavaScript, Flask, API RESTful, ProtoBUF \\
        
        Sécurité & Metasploit, Nmap, Nessus, ISO 2700x, Mehari \\
        
        Réseautage & TCP/IP, Modèle OSI, Wireshark, AWS \\
        
        Compétences Interpersonnelles & Excellente communication orale et écrite, méthodologies Agile et Scrum, JIRA, maîtrise de LaTeX(ce CV est realisé en LaTeX), Word, Excel, adaptation rapide à de nouveaux environnements, compétences aiguës en résolution de problèmes \\
        
        \end{tabularx}
    \end{rSection}
    \vspace{-0.75em}
    
    %----------------------------------------------------------------------------------------
    \begin{rSection}{Activités Extra-Scolaires} 
        \vspace{-0.25em}
        \begin{enumerate*}
            \item Leetcode et EulerProject pour le plaisir
            \item Passionné de lore Tolkien
            \item Amateur de cinéma
            \item SSIAP (emploi étudiant)
            \item Passionné par les Innovations Technologiques
        \end{enumerate*}
    \end{rSection}
\end{document}    
